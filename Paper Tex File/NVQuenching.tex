\documentclass[preprint,prl,twocolumn]{revtex4}

\usepackage{ORI_Group_style}
\usepackage{siunitx}


\graphicspath{{./figs/}}


\begin{document}

\title{Photophysics of Nitrogen Vacancy centres in Nanodiamonds}
  
\author{Reece P. Roberts$^{1,2}$}
\author{Author2$^{1,2}$}

\affiliation{$^1$ Department of Physics \& Astronomy, Macquarie University, NSW 2109, Australia}
\affiliation{$^2$ ARC Centre for Engineered Quantum Systems, Macquarie University, NSW 2109, Australia}


\begin{abstract}
I will write my abstract when I know my exact story.
The paper will be two papers, one for 780nm and one for 1042nm, unless I gain by putting the two papers together and build one higher impact paper.
The complete set of data should answer this question.
For example is there a difference between the mechanisms for each wavelength. For example the 780nm can both ionise and recombine, however maybe the 1042nm can only ionise and can't lease to a recombination process.
\end{abstract}

\maketitle

\section{Introduction}
The NV$^-$ charge state is an intensively studied material that has shown a wide range of applications in both Physics and Biology due to it's high stability and interesting optical properties. Biologists have used them extensively for biolabelling and imaging of internal biological structures (???ref3 and 4). Meanwhile, Physisits have been investigating their use in a wide range of nanoscale sensing and quantum sensing applications(???ref5). By exploiting the quantum mechanical interactions of the defects internal spin state, room temperature quantum effects can be observed in the NV$^-$ centre providing a platform to study a wide variety of quantum manipulation protocols (???ref6). However, these desirable effects rely solely on the properties arising from the NV$^-$ charge state and in most applications the excitation wavelength is chosen to be around 510-540nm, as this region was shown to have the highest charge state polarisation (ref7???). By using a single optimised excitation wavelength the impact of the neutral charge state NV$^0$ could be neglected despite the optimal charge state polarisation limited at $75\%$. 

In cases where only the single excitation laser is used the NV centre has been long stated to be extremely robust, with no bleaching or blinking under normal conditions(ref8???). However, in many cases once a second probe laser is used in an experiment the fluorescence of the NV centre is dramatically quenched(ref9???), preventing further applications and systems that require additional laser wavelengths. The quenching of fluorescence has been observed and described by numerous potential mechanisms, including (list here???1). In contrast to many of these mechanisms we are collecting the fluorescence of both charge states as well as probing in a non resonant continuous-wave regime of a few 10s of milliwatts eliminating many of the above mechanisms that rely transient mechanisms or high intensities fields. 

In this paper we investigate the quenching effects of the NV centre fluorescence in order to provide insight into the charge and spin state photo-dynamics. Our process is to measure the quenching dynamics of the NV centre under steady state illumination and using established physics of the NV centre in order to develop a rate equation model that describes the potential photo-physics of the system. Using this model various assumptions are analysed in order to determine their validity and the most likely model is determined by the Aikike information criteria. We believe this new rate equation model indicates the underlying physics that leads to the quenching of fluorescence that has been observed in the NV centre. By understanding the corresponding rates and processes we aim to apply particular initialisation processes to increase the spin and charge state polarisation of the NV centre. This will lead to direct enhancements of applications such as STED like imaging and for enhancing state preparation for NV based quantum technologies.




\todo{Should I add examples of further applications and systems that require additional laser wavelengths}


\section{Nitrogen Vacancy Centre}
Describe in more detail the centre itself
\subsection{Negatively Charge State}
Describe energy transitions and details of NV-
\subsection{Neutral Charged State}
Similarly do the same with NV0
\subsection{Ionisation Process}
Desribe NV- to NV0 Process
\subsection{Recombination Process}
Describe NV0 to NV- Process

\section{Experimental Details}
Describe the Quenching Experiment.

\section{Quenching Models}
In order to determine the intrinsic photophysics of our nanodiamonds we developed an 8 level rate equation model that incorporates both the ionisation and recombination mechanisms as well as the STED like mechanisms. The free paramaters of the model were varied in order to determine the most likely dynamics of the system. Four approaches were investigated and the most likely model was indetified by the Akaike information criteria blah blah

Insert simple diagram of the model and then for each model include separate rates.

\subsection{Underlying assumptions and unknowns}

\subsection{'STED' model}
Blah
\subsection{Simple Ionisation and Recombination Model}
Where the Ionisation and recombination rates are linearly dependant on laser power indepentant of wavelength.
\subsection{Wavelength Dependant I\&R}
Where the Ionisation and recombination rates are linearly dependant on laser power and dependant on wavelenght.
\subsection{Spin Depenatant I\&R}
Where the Ionisation and recombination rates are linearly dependant on laser power and dependant on wavelength. In addition there are separate ionisation rates from the $ms=\pm1$ and $ms=0$ of the excited NV- charge state to the ground state. However the ratio between these two ionisation channels is held constant for each laser wavelength.


\section{Akaike Information Criteria}
Blah Blah

Highlight the optimal model, Maybe put table here or in appendix.

\section{Discussion}
The optimal model is blah?

\subsection{Model Parameters}
The the values make sense?
What are the ionisation rates like? are the close to 10ms/mW. Spin dependancy may occur from different dipolar cross section between the two excited states.
Similar with the recombination rates? Link it to singlet nitrogen.
Recombination occurs into which ground state of the NV- 
Singlet recombination between 0.66 and 1 which is consistent.


\subsection{Charge state}
Blah
\subsection{Spin State}
Blah

\section{Comparison With Other Work}
Blah

\section{Conclusion}
Say that this indicates that this is the channels that are liekly to occur. If you want to use NV with other lasers than only the 532nm these dynamics must be understood. and then provide array of ways that could investigate these affects.
    


\section*{Acknowledgements}

....
%%%%%%%%%%%%%%%%%%%%%%%%%%%%%%%%%%%%%%%%%
%%%%%%%%%%%%%%%%%%%%%%%%%%%%%%%%%%%%%%%%%
%%%%%                                                                                                   %%%%%
%%%%%                                       APPENDIX                                          %%%%%
%%%%%                                                                                                   %%%%%
%%%%%%%%%%%%%%%%%%%%%%%%%%%%%%%%%%%%%%%%%
%%%%%%%%%%%%%%%%%%%%%%%%%%%%%%%%%%%%%%%%%

%
%\appendix
%
%\section{Experimental parameters}
%\label{param}
%




\begin{thebibliography}{27}






\end{thebibliography}




\end{document}
