\documentclass[preprint,prl,twocolumn]{revtex4}

\usepackage{ORI_Group_style}
\usepackage{siunitx}


\graphicspath{{./figs/}}


\begin{document}

\title{Photophysics of Nitrogen Vacancy centres in Nanodiamonds}
  
\author{Reece P. Roberts$^{1,2}$}
\author{Author2$^{1,2}$}

\affiliation{$^1$ Department of Physics \& Astronomy, Macquarie University, NSW 2109, Australia}
\affiliation{$^2$ ARC Centre for Engineered Quantum Systems, Macquarie University, NSW 2109, Australia}


\begin{abstract}
I will write my abstract when I know my exact story.
The paper will be two papers, one for 780nm and one for 1042nm, unless I gain by putting the two papers together and build one higher impact paper.
The complete set of data should answer this question.
For example is there a difference between the mechanisms for each wavelength. For example the 780nm can both ionise and recombine, however maybe the 1042nm can only ionise and can't lease to a recombination process.
\end{abstract}

\maketitle

\section{Introduction}
Introduce NV what it is used for, what people want to use it for, trying to integrate into other systems. But really it it still not very well understood as there have been many observed quenching mechanisms. 

\section{Nitrogen Vacancy Centre}
Describe in more detail the centre itself
\subsection{Negatively Charge State}
Describe energy transitions and details of NV-
\subsection{Neutral Charged State}
Similarly do the same with NV0
\subsection{Ionisation Process}
Desribe NV- to NV0 Process
\subsection{Recombination Process}
Describe NV0 to NV- Process

\section{Experimental Details}
Describe the Quenching Experiment.

\section{Quenching Models}
In order to determine the intrinsic photophysics of our nanodiamonds we developed an 8 level rate equation model that incorporates both the ionisation and recombination mechanisms as well as the STED like mechanisms. The free paramaters of the model were varied in order to determine the most likely dynamics of the system. Four approaches were investigated and the most likely model was indetified by the Akaike information criteria blah blah

Insert simple diagram of the model and then for each model include separate rates.

\subsection{Underlying assumptions and unknowns}

\subsection{'STED' model}
Blah
\subsection{Simple Ionisation and Recombination Model}
Where the Ionisation and recombination rates are linearly dependant on laser power indepentant of wavelength.
\subsection{Wavelength Dependant I\&R}
Where the Ionisation and recombination rates are linearly dependant on laser power and dependant on wavelenght.
\subsection{Spin Depenatant I\&R}
Where the Ionisation and recombination rates are linearly dependant on laser power and dependant on wavelength. In addition there are separate ionisation rates from the $ms=\pm1$ and $ms=0$ of the excited NV- charge state to the ground state. However the ratio between these two ionisation channels is held constant for each laser wavelength.


\section{Akaike Information Criteria}
Blah Blah

Highlight the optimal model, Maybe put table here or in appendix.

\section{Discussion}
The optimal model is blah?

\subsection{Model Parameters}
The the values make sense?
What are the ionisation rates like? are the close to 10ms/mW. Spin dependancy may occur from different dipolar cross section between the two excited states.
Similar with the recombination rates? Link it to singlet nitrogen.
Recombination occurs into which ground state of the NV- 
Singlet recombination between 0.66 and 1 which is consistent.


\subsection{Charge state}
Blah
\subsection{Spin State}
Blah

\section{Comparison With Other Work}
Blah

\section{Conclusion}
Say that this indicates that this is the channels that are liekly to occur. If you want to use NV with other lasers than only the 532nm these dynamics must be understood. and then provide array of ways that could investigate these affects.
    


\section*{Acknowledgements}

....
%%%%%%%%%%%%%%%%%%%%%%%%%%%%%%%%%%%%%%%%%
%%%%%%%%%%%%%%%%%%%%%%%%%%%%%%%%%%%%%%%%%
%%%%%                                                                                                   %%%%%
%%%%%                                       APPENDIX                                          %%%%%
%%%%%                                                                                                   %%%%%
%%%%%%%%%%%%%%%%%%%%%%%%%%%%%%%%%%%%%%%%%
%%%%%%%%%%%%%%%%%%%%%%%%%%%%%%%%%%%%%%%%%

%
%\appendix
%
%\section{Experimental parameters}
%\label{param}
%




\begin{thebibliography}{27}






\end{thebibliography}




\end{document}
