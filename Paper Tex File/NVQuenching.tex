\documentclass[,prl,twocolumn]{revtex4}

\usepackage{ORI_Group_style}
\usepackage{siunitx}


\graphicspath{{./figs/}}


\begin{document}

\title{Photophysics of Nitrogen Vacancy centres in Nanodiamonds}
  
\author{Reece P. Roberts$^{1,2}$}
\author{Author2$^{1,2}$}

\affiliation{$^1$ Department of Physics \& Astronomy, Macquarie University, NSW 2109, Australia}
\affiliation{$^2$ ARC Centre for Engineered Quantum Systems, Macquarie University, NSW 2109, Australia}


\begin{abstract}
NVs are cool because they do cool sciency stuff.
All cool stuff comes from NV-.
People use 532 excitation to increase charge polarisation so they can optimise cool stuff and neglect NV0.
They looked at it with only one excitation wavelength at a time. 
Clearly charge polarisation depends on wavelength.
So clearly quenching must occur with second wavelength.
Must understand all of this stuff to intergrate NVs into other systems that use lasers.
\end{abstract}

\maketitle

\section{Introduction}
The Nitrogen Vacancy (NV) centre is is a point defect consisting of a nitrogen-vacancy lattice pair embedded along the [111] axis of a diamond (???ref1). The NV centre has two stable charge states, the neutral charge state (NV$^0$) and the negatively charged state (NV$^-$), with photo-induced interconversion between these two states (???ref2). The NV$^-$ charge state is an intensively studied material that has shown a wide range of applications in both Physics and Biology due to it's high stability and interesting optical properties. Biologists have used them extensively for biolabelling and imaging of internal biological structures (???ref3 and 4). Meanwhile, Physisits have been investigating their use in a wide range of nanoscale sensing and quantum sensing applications(???ref5). By exploiting the quantum mechanical interactions of the defects internal spin state, room temperature quantum effects can be observed in the NV$^-$ centre providing a platform to study a wide variety of quantum manipulation protocols (???ref6). However, these desirable effects rely solely on the properties arising from the NV$^-$ charge state and in most applications the excitation wavelength is chosen to be around 510-540nm, as this region was shown to have the highest charge state polarisation (ref7???). By using a single optimised excitation wavelength the impact of the neutral charge state NV$^0$ could be neglected despite the optimal charge state polarisation limited at $75\%$. 

In cases where only the single excitation laser is used the NV centre has been long stated to be extremely robust, with no bleaching or blinking under normal conditions(ref8???). However, in many cases once a second probe laser is used in an experiment the fluorescence of the NV centre is dramatically quenched(ref9???), preventing further applications and systems that require additional laser wavelengths. The quenching of fluorescence has been observed and described by numerous potential mechanisms, including (list here???1). In contrast to many of these mechanisms we are collecting the fluorescence of both charge states as well as probing in a non resonant continuous-wave regime of a few 10s of milliwatts eliminating many of the above mechanisms that rely transient mechanisms or high intensities fields. 

In this paper we investigate the quenching effects of the NV centre fluorescence in order to provide insight into the charge and spin state photo-dynamics. Our process is to measure the quenching dynamics of the NV centre under steady state illumination and using established physics of the NV centre in order to develop a rate equation model that describes the potential photo-physics of the system. Using this model various assumptions are analysed in order to determine their validity and the most likely model is determined by the Aikike information criteria. We believe this new rate equation model indicates the underlying physics that leads to the quenching of fluorescence that has been observed in the NV centre. By understanding the corresponding rates and processes we aim to apply particular initialisation processes to increase the spin and charge state polarisation of the NV centre. This will lead to direct enhancements of applications such as STED like imaging and for enhancing state preparation for NV based quantum technologies.

\todo{I should also include some of our conclusions, Don't bury the lede!}
\todo{Should I add examples of further applications and systems that require additional laser wavelengths}

\subsection{Negatively Charge State}
The energy level diagram of the negatively charged NV$^-$ centre  can be observed in Fig. \ref{FigEnergyLevelsNV-}.

\begin{figure}[H]
  \centering
  \includegraphics[width=0.4\textwidth]{NV-.jpg} 
 \caption{Energy levels of the NV$^-$ centre.} \label{FigEnergyLevelsNV-}
\end{figure}

The NV$^-$ consists of a ground triplet state $^3A_2$ and an excited triplet state $^3E$, as well as two metastable singlet states $^1E_1$ and $^1A_1$ ref11???. Within the spin triplet states the $m_s = 0$ ans $m_s = \pm1$ spin states are split in energy at zero magnetic field by $D=\SI{2.87}{GHz}$ for the ground triplet and $D=\SI{1.42}{GHz}$ for the excited triplet (Ref12???). The spin transition rate between the ground $m_s = 0$ ans $m_s = \pm1$ spin states is given by the spin-lattice relaxation time $T_1$ and has been measure to be $\approx\SI{6}{ms}$ at room temperature and zero magnetic field (ref13???). This spin mixing rate can be dramatically increased by appling a large magnetic field in the vicinity of the NV centre ref14???.

\todo{I am being vague about the spin mixing rate because because I believe the current explanation of Zeeman splitting the spin sublevels has holes in it. Ask me for more info or look at comment in tex file}
% This explanation would not explain the saturation effect that are observed as you bring in the magnet. Following the Zeeman splitting logic you would expect only one spot where there is large mixing and then you would reduce the mixing as you cross this precise field strength. However this saturation effect may be a result of having a collection of NV centres, however Xavi and Matt observed the same effect with single NV centres. This leads to the belief that the spin mixing may not be due to zeeman splitting and the absolute magnitude of the magnetic field but instead due to the thermal perturbations of the magnet destroying the spin coherence. In any case someone familiar with NV centres will accept this statement anyway. Although the question of why there is no spin mixing on the excited state is still a valid question that I still have no answer for other than it is completely absent from the literature and at this stage (22/12/16) my model does indicated that it does not occur. 	I also had the thought that the spin mixing rate is still increasing as we are bringing in the magnet but the rate is already high enough to spin depolarise the system. This can be checked using the model.
%If they were only capturing NV- or mostly NV- (I Never got a good answer to this question from MAtt and Xavi) then due to a combination of the magnet mixing the spin state, which will give an \~\%30 quenching by it self, and then due to spin dependant ionisation this mixed spin state will lead to a higher Ionisation rate and hence a reduction in NV$^-$ charge state polarisation and reduction in Fuorescence.
 
The principle zero-phonon line between the $^3A_2$ and $^3E$ is centred at 637nm and can be efficiently excited with spin conservation at most wavelengths below 640nm (ref15???). The radiative lifetime of the excited state is $\SI{13}{ns}$ for NV centres in bulk diamond (ref16???) and approximately $\SI{25}{ns}$ for NV centres in nanodiamonds (ref17???). Only a few percent of the Fluorescence is emitted at the ZPL, most fluorescence appears in the phonon side bands between 600 and 800nm as shown in Fig. \ref{FigFluoro}.

\begin{figure}[t]
  \centering
  \includegraphics[width=0.4\textwidth]{Fluoro.png} 
 \caption{ Fluoresence profiles of NV charge states .} \label{FigFluoro}
\end{figure}

The excited triplet states can also decay to the excited singlet state, the rate from the $m_s=\pm1$ excited triplet state is $2\pi\times9.4\times10^6\SI{}{GHz} = \SI{16.9}{ns}$, whereas the rate from the $m_s=0$ excited triplet state is almost an order of magnitude smaller at $ 2\pi\times1.8\times10^6 \SI{}{GHz} = \SI{88.4}{ns}$ (ref18???). Whilst it not completely understood why there is a large discrepancy between these decay channels it is noteworthy that this discrepancy is what leads to many of the interesting optical properties of the NV centre. It causes a difference in fluorescence intensity between the two excited spin states which in turn leads to a mechanism for an all optical readout of the centres internal spin state. The excited singlet state has a lifetime of $\approx\si{1}{ns}$ (ref19???) and populates the longer lived ground singlet state and it has been shown to emit fluorescence at a ZPL of $\SI{1042}{nm}$ ref20???. The longer lived ground metastable state has a lifetime of $\SI{150}{ns}$ and decays into the ground the triplet spin state ref21???. It was commonly believed that this population decayed only into the $m_s=0$ spin state, however a recently this is being challenged and it has been claimed that the decay into the ground triplet has a spin state ratio closer to $\frac{m_s=0}{m_s=\pm1} = 1.1-2$ ref22???. 

\subsection{Neutral Charged State}
As opposed to the rigorous study the NV$^-$ charge state has received the $NV^0$ charge state has often been neglected. However in order to study the NV centre as a whole it must be included. We use the established three level model to describe its intrinsic dynamics. The energy level diagram of the neutral charged NV$^0$ centre  can be observed in Fig. \ref{FigEnergyLevelsNV0}.

\begin{figure}[H]
  \centering
  \includegraphics[width=0.4\textwidth]{NV0.jpg} 
 \caption{Energy levels of the NV$^-$ centre.} \label{FigEnergyLevelsNV-}
\end{figure}

The NV$^0$ charge state consists of of a ground doublet $^2E$ and an excited doublet $^2A$ with a ZPL at $\SI{575}{nm} = \SI{2.156}{eV}$.  It can be efficiently excited at most wavelengths below $\SI{675}{nm}$ ref23??? and has a radiative lifetime of $\Gamma_{e^Zg^Z} = 19\times2 \SI{}{ns}$ ref26???. The exact excitation cross section is unknown, however, the ratio of excitation cross sections between NV$^0$ and NV$^-$ can be measured by looking at their relative emission intensities. In addition, since the quantum efficiency of NV$^0$ and NV$^-$ is $\approx 1$ then the ratio of excitation cross sections is given by the ratio of emission cross sections giving $\Gamma_{g^Ze^Z} = \frac{1}{3} \Gamma_{g^Te^T}$ ref25???. Differing from NV$^-$, NV$^0$ does not have detectable magnetic resonances associated with its degenerate spin doublet ground and excited states ref24???. Only a few percent of the Fluorescence is emitted in the ZPL, whereas most fluorescence appears in the phonon side bands between 550 and 750nm. However, no ODMR or optical readout of this metastable quartet have been measured and it is expected to have negligible impact on the photo-physics of the NV centre and as such has been neglected from our analysis. \todo{This statement was true as of February 2013, I need to ensure that this is still true.}
  
\todo{I think I doubled this value because NV- lifetimes were doubled in ND as compared to bulk and the lifetime I found was in Bulk diamond. Either need to find value for ND's or argue that it should also increase the same way as NV- did...}

\subsection{Ionisation \& Recombination}
To convert between the two charge states we need to examine both the Ionisation process from NV$^-$ to NV$^0$ and the recombination process from NV$^0$ to NV$^-$. The desirable effects of the NV centre rely solely on the properties arising from the NV$^-$ charge state and as a result a standard $\approx \SI{532}{nm}$ excitation laser used in NV centre applications is chosen so that it produces the highest charge state polarisation in an effort to optimise the effect and allow any effects due to the $NV^0$ charge state to be neglected (ref7???). However, any additional laser is going to alter this maximised charge state polarisation. 

\subsection{Ionisation Process}
Ionisation from NV- to NV0 occurs in a two step process as shown in Fig. \ref{FigChargeConversiona}.

\begin{figure}[H]
  \centering
  \includegraphics[width=0.4\textwidth]{ChargeConversiona.png} 
 \caption{\textbf{Ionsation Processes.} \textbf{a,} Ionisation from NV$^-$ to NV$^0$ electron view \textbf{b,} Ionisation pathways traditional view.} \label{FigChargeConversiona}
\end{figure}

First a photon must excite an electron into the excited $^3E$ state of the NV$^-$. The electron can then be excited again into the conduction band leading to an Auger ionisation process which strips an electron from the centre converting it into the NV$^0$ charge state in its ground state configuration (ref28???). This two step process has only been investigated with a single excitation laser which leads to an ionisation rate that is quadratic with excitation power and can no longer occur at wavelength greater than the ZPL of the transition. However, this process can be mediated by two lasers, one that strongly excites the transition and one that strongly ionises the electron leading to the Auger ionisation process.

\subsection{Recombination Process}
The recombination process from NV$^0$ to NV$^-$ also occurs in a two step process which is shown in Fig. \ref{FigChargeConversionb}.

\begin{figure}[H]
  \centering
  \includegraphics[width=0.4\textwidth]{ChargeConversionb.png} 
 \caption{\textbf{Recombination Processes.} \textbf{a,} Recombination from NV$^-$ to NV$^0$ electron view \textbf{b,} Recombination pathways traditional view.} \label{FigChargeConversionb}
\end{figure}

First a photon must excite an electron in the NV$^0$ charge state into the excited $^2A$ state. A second photon can then be excited from the valence band into the $^2E$ ground state which provides the extra electron to the centre converting it into the NV$^-$ charge state in its ground state configuration (ref29???). Currently there is no evidence to indicate which spin state the NV$^-$ charge state will now be populated in, however it has recently been observed that the ionisation, recombination process is a spin depolarising process indicating a non negligible component in the $m_s=\pm1$ spin state ref30???.

Whilst the ionisation and recombination processes have been investigated for single excitation wavelengths we believe however that one can excite the transition with a wavelength $<\SI{575}{nm}$ satisfying the first stage of the recombination process and then promote electrons from the valance band with a wavelength longer than $\SI{575}{nm}$. It has been proposed also that the conversion from NV$^0$ to NV$^-$ is mediated by ionisation of single substitutional Nitrogen impurities ($N_s$) in the nanodiamonds providing free electrons to combine with the NV$^0$ charge state ref32???. Ionisation of $N_s$ impurities requires $>\SI{1.7}{eV}$ for bulk diamond and slightly lower energy $>\SI{1.6}{eV}$ for nanodiamonds ref33???. Our nanodiamonds are a highly irradiated sample and therefore contain a high concentration of single substitutional nitrogen $N_s$. We postulate that the quenching observed in our nanodiamonds could be due to a dramatic increase in the ionisation and recombination rates induced by the NIR lasers and developed a rate equation model to determine the likelihood of this process compared to a STED like mechanism.


\section{Experimental Details}
In our experiment, (???2) type and size nanodiamonds are dispersed on a glass coverslip placed on the sample plane of a custom built scanning confocal microscope. The NV centres are pumped with a 532nm continuous wave laser after focusing through a x.xNA, Brand, type immersion objective lens ???3. The 780nm laser is combined and then superimposed with the 532nm laser before the objective lens. The fluorescence is collected through the same objective and sent to an avalanche photodiode (APD) that collects all wavelengths between $550-\SI{700}{nm}$. A permanent neodymium magnet was placed on a moveable arm above the sample plane so that a large non zero magnetic field could be brought in close proximity to the nanodiamond in order to investigate the effect of mixing the spin state of the NV$^-$ ground states. The setup is shown in Fig. \ref{FigSetup}.

\begin{figure}[t]
  \centering
  \includegraphics[width=0.3\textwidth]{Setup.png} 
 \caption{\textbf{Experimental approach.} This is the setup ???} \label{FigSetup}
\end{figure}

For each nanodiamond the saturation curve of the fluorescence of the NV centre is measured. The power dependance of the flourescence for the $\SI{780}{nm}$ illumination wavelength is then measured for five powers of the $\SI{532}{nm}$ excitation laser. The neodymium magnet is placed $\approx \SI{0.5}{mm}$ above the sample plane of the confocal microscope in order to mix the spin state of the NV- charge state. The above set of measurements are then repeated to provide more information on the internal spin state of the system. For each nanodiamond we get a series of plots that can be observed in figure \ref{FigData}.


\begin{figure}[H]
  \centering
  \includegraphics[width=0.4\textwidth]{Data2.png} 
 \caption{\textbf{Experimental data.} \textbf{a.} This is the data swap a and b???} \label{FigData}
\end{figure}

\todo{Describe what the data is and what it indicates.}
\todo{I many also want to use a magnet and no magnet trace in one window and show that for high quenching they are spin depolarised.}

\section{Quenching Models}
In order to determine the intrinsic photophysics of our nanodiamonds we developed an 8 level rate equation model that incorporates both the ionisation and recombination mechanisms as well as the STED like mechanisms. The free parameters of the model were varied in order to determine the most likely dynamics of the system. Six submodels were investigated and the most likely model was identified by the Akaike information criteria. The six models can be seen in figure \ref{FigModels}.

\begin{figure}[H]
  \centering
  \includegraphics[width=0.5\textwidth]{models.png} 
 \caption{\textbf{Experimental models.} \textbf{a.} This are the six models we looked at} \label{FigModels}
\end{figure}


The models can separated by their underlying quenching mechanism whether it be a STED like process, an Ionisation and Recombination process or a combination of the two. These three groups can then be split in two depending on if the model is a spin dependent quenching model or a spin independent quenching model. Below is a short description of each of the separate model types, for a comprehensive description of the rate equations and models refer to the supplementary material.

\subsubsection{STED like models}
The STED like models are models that describe the quenching of the fluorescence through a true stimulated emission process or even by a  processes that are mediated though an extra metastable fast decay state of the NV$^-$ centre ref???.

\subsubsection{Ionisation and Recombination models}
The ionisation and recombination models relies on the two step processes that convert the charge state of the NV centre. Each time the NV centre is excited the flouresence of the NV centre is reduced due to the increased non radiative decay channels enabled by the conversion of charge state of the NV centre.

\subsubsection{Spin Dependancy}
In each of the above mechanisms the quenching of the fluorescence from the NV$^-$ charge state can occur from either the $m_s=0$ or $m_s=\pm1$ excited state. One would initially assume that the ionisation and STED like mechanisms would be spin independant, however these energy levels already show a spin dependant nature towards the siglet states. 
\todo{Maybe correct need to find proof, maybe rochelles thesis or superradiance paper: In addition the dipole strength of the excited $m_s=\pm1$ state in nanodiamonds containing a collections of NV centres is $\approx 10\times$ that of the dipole strength of the excited $m_s=0$ state.}
As a result we investigated the possibility that the interactional cross sections of the excited NV$^-$ centre can be spin dependant.


\section{Akaike Information Criteria}
What is the Aikike infromation criteria.
Which model is the best model?

\section{Discussion}
The optimal model is a spin dependant Ionisation and recombination model.

\subsection{Model Parameters}
The the values make sense?
What are the ionisation rates like? are the close to 10ms/mW. Spin dependency may occur from different dipolar cross section between the two excited states.
Similar with the recombination rates? Link it to singlet nitrogen.
Recombination occurs into which ground state of the NV- 
Singlet recombination between 0.66 and 1 which is consistent.


\subsection{Charge state}
Blah
\subsection{Spin State}
Blah

\section{Comparison With Other Work}
Blah

\section{Conclusion}
Say that this indicates that this is the channels that are liekly to occur. If you want to use NV with other lasers than only the 532nm these dynamics must be understood. and then provide array of ways that could investigate these affects.
    


\section*{Acknowledgements}

....
%%%%%%%%%%%%%%%%%%%%%%%%%%%%%%%%%%%%%%%%%
%%%%%%%%%%%%%%%%%%%%%%%%%%%%%%%%%%%%%%%%%
%%%%%                                                                                                   %%%%%
%%%%%                                       APPENDIX                                          %%%%%
%%%%%                                                                                                   %%%%%
%%%%%%%%%%%%%%%%%%%%%%%%%%%%%%%%%%%%%%%%%
%%%%%%%%%%%%%%%%%%%%%%%%%%%%%%%%%%%%%%%%%

%
%\appendix
%
%\section{Experimental parameters}
%\label{param}
%




\begin{thebibliography}{27}






\end{thebibliography}




\end{document}
